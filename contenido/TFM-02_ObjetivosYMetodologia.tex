\chapter{Objetivos y Metodología}\label{chap:objetivos}
\textbf{Objetivos concretos y metodología de trabajo}

%Párrafo introductorio del capítulo
[Este bloque es el puente entre el estudio del dominio y la contribución a realizar. Según el tipo concreto de trabajo, el bloque se puede organizar de distintas formas, pero los siguientes elementos deberían estar presentes con mayor o menor detalle.]

%--- General Objectives ---%
\section{Objetivo General}\label{sec:objgeneral}
El objetivo general del presente trabajo es extender la capacidad de monitorización que se tiene sobre una red empresarial.
Se enfocará para ello en la detección de anomalías, haciendo uso de técnicas basadas en los equipos que la componen (a diferencia de las basadas en flujos de tráfico).
De este modo, se desarrollará un sistema que categorice mediante clustering los equipos finales de la red
y que pueda revelar cuándo la actividad de un equipo se desvía de su comportamiento habitual.

Con ello se espera seguir mejorando las prestaciones ofrecidas, especialmente en seguridad, pero también ayudará en la identificación de problemas de configuración, cambios de tendencia en la red y en definitiva cualquier evento inesperado que pudiera repercutir negativamente en la operativa normal de la empresa monitorizada.

%--- Specific Objectives ---%
\section{Objetivos Específicos}\label{sec:objespecificos}
La consecución de dicho objetivo general vendrá pautada por las siguientes metas más específicas, que se abordarán en este orden:

\begin{itemize}

\item Determinar qué características son las más relevantes a la hora de clusterizar la red y extraerlas en tiempo real.

\item Establecer unas categorías básicas mediante clasificación no supervisada, que sean verificables con la documentación que se tiene de la red.

\item Comprobar que el sistema detecta anomalías si se le proporcionan datos modificados intencionadamente.

\item Refinar el algoritmo de clustering empleado, buscando categorías más específicas que no se estuvieran teniendo en cuenta antes.

\item Diseñar una forma de visualización adecuada para los resultados.

\item Alcanzar un modo de funcionamiento en tiempo real, en coordinación con los demás mecanismos de monitorización presentes y asegurando una precisión razonable (detecciones correctas frente a falsos positivos).

\end{itemize}

%--- Work Methodology ---%
\section{Metodología de Trabajo}\label{sec:metodología}

[Sección previsiblemente modificable y ampliable según avance en el desarrollo del trabajo]

Con los anteriores objetivos específicos marcados, se hace necesario establecer un marco de trabajo para llevarlos a cabo.
En esta sección se estructurará una metodología de trabajo en base a cuatro fases generales, que tienen como fin último alcanzar el objetivo general.
Sus enunciados vendrán acompañados de una descripción, explicando con un poco más de detalle cómo se planea ejecutar cada etapa.

\begin{itemize}

    \item Selección de las características más relevantes para la clasificación

A partir de la heterogénea colección de logs de varios firewalls y eventos de antivirus de equipos Windows de la que se dispone, se identificarán los datos que puedan ser interesantes para el clustering.
Se tendrán en cuenta distintos aspectos que respondan a las actividades que realiza un equipo informático en una red empresarial, como por ejemplo el uso de determinados servicios, las visitas a sitios web de una serie de categorías, horarios de actividad, interacción con ciertos servidores, etc.
A continuación, se analizarán estas características a través de técnicas estadísticas, gracias a las cuales se adquirirá un entendimiento preliminar de la importancia que cada una supone para la posterior clasificación de instancias.
Es posible que se logre la reducción de su dimensionalidad, la eliminación de alguna característica redundante o en definitiva alguna simplificación que haga el clustering más sencillo de ejecutar.

    \item Obtención de categorías mediante clustering

Se pasará entonces a aplicar por primera vez algunos algoritmos de clustering sobre estos datos.
Inicialmente se elegirán dos algoritmos simples y ampliamente conocidos como son k-means y DB-SCAN.
El primero es un algoritmo exclusivo basado en centroides, mientras que el segundo particiona según la densidad de los clusters formados.
Se experimentará con diferentes valores para sus parámetros, de forma que puedan compararse resultados.
De cara a determinar la efectividad de los algoritmos, se considerarán medidas objetivas que representen su rendimiento.
Los métodos típicos para ello valoran características intrínsecas o derivadas como su complejidad, estabilidad y tiempo de computación, así como métricas de validación internas y externas como la silueta de sus clusters o el índice de Rand.
Estas medidas son útiles para evaluar la calidad de resultados del algoritmo, pero debe apreciarse que proporcionan un escaso conocimiento sobre lo que contienen los clusters.
Además, se emplearán los indicadores comunes con los que se suele definir la bondad de una técnica de clasificación en aprendizaje automático.

    \item Detección de intrusiones con análisis del clustering

Una vez se cuente con una clasificación satisfactoria, se procederá a la siguiente fase, en la cual se pretende detectar anomalías en los datos.
En caso de contar con instancias que se hayan identificado como intrusiones mediante otros métodos, se usarán para probar la capacidad de detección del sistema.
Si, llegado el momento de hacer este testeo, no se dispone de este tipo de ejemplos, se elaborarán datos que representen distintas clases de anomalías con diversos grados de evidencia.
Dado que también se desea detectar otras clases de anomalías más genéricas (no solo relacionadas con seguridad), se incluirán casos como cambios de configuración o de equipos que hayan cambiado de rol en la red.

    \item Evaluación en escenario real

Finalmente, se desplegará el sistema desarrollado en un entorno de producción.
Se integrará con el framework de alerting empleado y se ofrecerá una representación visual adecuada que permita al analista aprovechar el valor que el sistema aportará a la hora de revisar casos alertados.
Además, se automatizará la extracción de características y se prepararán el resto de componentes del sistema para que funcione en tiempo real.

\end{itemize}

[Se informará del cumplimiento adecuado del Reglamento General de Protección de Datos]
